%% ----------------------------------------------------------------------------
%% John's LaTex package and environment setup - Not document type specific]
%% V1.5
%% 1 Sep 2016
%% ----------------------------------------------------------------------------
% John Fogarty
% Web:    http://www.jfogarty.org
% E-Mail: johnhenryfogarty@gmail.com
% --------------------------------------------------------------------------
% Copyright 2015-2016 John Fogarty
% 
% This work may be distributed and/or modified under the conditions of the
% LaTeX Project Public License, either version 1.3 of this license or
% (at your option) any later version.
% The latest version of this license is in
%      http://www.latex-project.org/lppl.txt
% and version 1.3 or later is part of all distributions of 
%      LaTeX version 2005/12/01 or later.
% --------------------------------------------------------------------------
\usepackage{etoolbox}                   % Extended macro management tools
\usepackage{xparse}                     % Extended newcommands
\usepackage{makeidx}					% Enable Index creation
\usepackage[tight]{shorttoc}            % Allow an abbreviated table of contents.

% Note BIBLATEX is *cannot* be enabled with these packages.
%\usepackage[backend=biber]{biblatex}   % Use the BIBER citation generator
%\usepackage[backref=true]{biblatex}    % Bibliography with back hyperlink.

\ifthenelse{\boolean{NOAUTHORS}}{}{
  \usepackage[biblabels]{authorindex}   % Enable Authors Cite list creation
}
\usepackage{multicol}                   % Multiple columns for the index
\usepackage{blindtext}                  % Nonsense text fillers
\usepackage{ifxetex,ifluatex}           % processor conditionals
\usepackage{wrapfig}                    % Support wrapfigure environment
\usepackage{endnotes}                   % Support endnotes.
\usepackage{titling}                    % fancy titles with prefix and suffix

\usepackage[T1]{fontenc} 				% enables special graphics
\usepackage{textcomp}    				% LaTex fonts and glyph encodings such as \textonehalf
\usepackage{lmodern}    				% LaTex modern math
\usepackage[utf8]{inputenc}             % allows limited support for non-ascii UTF8 codes in text
\usepackage{graphicx}                   % graphics driver support (math/images for PDFs)

% XXX %\usepackage[parfill]{parskip} 		% Seperate paragraphs with spaces, no indent.
% \usepackage{parskip}                    % Separate paragraphs by a blank line.
\usepackage[english]{babel}             % Support for non-english typesetting
\usepackage[autostyle, english = american]{csquotes} % Backquotes
% XXX %\usepackage{titlesec}                % title, chapter and section formatting
\usepackage{enumerate}					% styles enumerated lists

\usepackage{changepage}				    % page and margin adjustments

% Note ~/.config/texstudio/subcaption_jf.cwl file needed for autocompletion in TexStudio.
\ifthenelse{\boolean{SUBCAPTIONS}}{
  \usepackage{subcaption}               % subcaptions in figures]
}{}

% Math and Software 
%\usepackage{amsmath}   			    % American math (cannot use with wasysym)
\usepackage{wasysym}        			% Additional symbols
\usepackage{amssymb}        			% American math symbols
\usepackage{latexsym}       			% LaTex symbols
%\usepackage{amsfonts}

%\usepackage{fontspec}    % -- Use with XeLaTex only for OpenType fonts
%\usepackage{lilyglyphs}  % -- Use with XeLaTex only, Musical notations

%\usepackage{listings}					% Computer language source code typesetting
%\usepackage{tikz}						% 3D plots - Defined in Parameters.
%\usepackage{pgfplots}                  % Complex graphs, charts, and plots

%\usepackage[nottoc]{tocbibind}         % Place the References, Index in TOC
%\usepackage[makeindex]{imakeidx}        % Make index hyperref to body
% Urls and Hrefs that work in PDFs [should be last] package.
\usepackage[hyphens]{url}  

\ifthenelse{\boolean{NOHYPERREF}}{
  \newcommand\hypersetup[1]{}%
  \newcommand{\href}[2]{Href=#1}%
  \newcommand{\nameref}[1]{NameRef=#1}%
}{%
  \ifthenelse{\boolean{TOHTML}}{
    \usepackage[plainpages=false]{hyperref}
  }{
    \usepackage[pdftex,
            pdfauthor={\TheAuthor},
            pdftitle={\TheMainTitle: \TheSubTitle},
            pdfsubject={\TheSubjectArea},
            pdfkeywords={\TheKeywords},
            plainpages=false]{hyperref}
  }
}

\usepackage{xspace}                     % Useful in parameterless macros

%--------------------------------------------------------------------
% controls the TOC, Footnote, and URL/Hyperref link colors.
\colorlet{HtmlUrlColor}{green!22!black}% The color for named HREFs/URLs
\colorlet{HtmlCiteColor}{green!50!black}% The color for citations
\colorlet{HtmlLinkColor}{blue!70!black}% The color for in text links

\ifthenelse{\boolean{TOPRESS}}{
  \hypersetup{colorlinks=true, citecolor=black, filecolor=black, linkcolor=black, urlcolor=black}%
}{
  \ifthenelse{\boolean{NOCOLOR}}{
    \hypersetup{colorlinks=true, citecolor=black, filecolor=black, linkcolor=black, urlcolor=black}
  }{
    \ifthenelse{\boolean{TOHTML}}{
      \hypersetup{colorlinks=true,
          citecolor=HtmlCiteColor,
          filecolor=green,
          linkcolor=HtmlLinkColor,
          urlcolor=HtmlUrlColor}
    }{ 
      \hypersetup{colorlinks=true, linkcolor=black, urlcolor=blue}
    }
  }
}

%%---------------------------------------------------------------------
%% Hyperlinks removed on PRINT edition.
\ifthenelse{\boolean{TOPRESS}}{%
  \newcommand{\HREF}[2]{\textsf{#1} [\allowbreak{\detokenize{#2}}]}%
  \newcommand{\HREFX}[3]{\textsf{#1} [#3]}%
  \newcommand{\URL}[1]{[{\allowbreak{\detokenize{#1}}}]\xspace}%
  \newcommand{\NameRef}[2][] {[see:{#2}]}
  \newcommand{\AMAZON}[1]{}
}{%
  \ifthenelse{\boolean{RAWHTML}}{%
    \newcommand{\HREF}[2]{\HCode{<a href="#2" >}\textsf{#1}\HCode{</a>}\xspace}%
    \newcommand{\HREFX}[3]{\HCode{<a href="#2" >}\textsf{#1}\HCode{</a>}\xspace}%
  }{%
    \newcommand{\HREF}[2]{\textsf{\href{#2}{#1}}\xspace}%
    \newcommand{\HREFX}[3]{\textsf{\href{#2}{#1}}\xspace}%
  }%
  \newcommand{\URL}[1]{\url{#1}\xspace}%
  \newcommand{\NameRef}[2][]{[see:\nameref{#2}]}%
  \newcommand{\AMAZON}[1]{\HREF{[@Amazon]}{#1}}
}

%%---------------------------------------------------------------------
%% Index generation is optional.
\ifthenelse{\boolean{NOINDEX}}{%
\newcommand\Index[1]{}%
\newcommand\PrintIndex{}%
}{%
\newcommand\Index[1]{\index{#1}}%
\newcommand\PrintIndex{\printindex}%
}%

%---------------------------------------------------------------------
% Add Index[optional index text]{permuted section name}{index text}
\newcommand{\CatIndex}[2]{\Index{#1!\textsf{#2}}}

\newcommand{\Ix}[3][]{%
  \ifthenelse{\isempty{#1}}{\CatIndex{#2}{#3}\Index{#3}}{\CatIndex{#2}{#1}\Index{#1}}%
}

\newcommand{\Inx}[2][]{\ifthenelse{\isempty{#1}}{#2\Index{#2}}{#2\Index{#1}}}
\newcommand{\PInx}[3][]{#3\Ix[#1]{#2}{#3}}

%%---------------------------------------------------------------------
\newcommand{\ParagraphIndent}[1]{%
	\setlength\parindent{#1ex}%
}

\newcommand{\Indent}{%
	\ParagraphIndent{4}%
}

\newcommand{\StopIndent}{%
	\ParagraphIndent{0}%
}

%%---------------------------------------------------------------------
% Defines the heading text for a chapter's notes.
\newcommand{\ChapterNotesName}[1]{%
	\ifthenelse{\boolean{CHAPTERNOTES}}{%
		\renewcommand{\notesname}{Notes [#1]}%
	}{%
		\addtoendnotes{
			\bigskip\StopIndent
			\large\textsf{Chapter \thechapter\, --- #1}
			\normalsize\normalfont\bigskip
		}
	}%    
}

%%---------------------------------------------------------------------
% After each chapter reset the note numbering and output the 
% accumulated chapter notes if requested.
\newcommand{\ChapterSuffix}{%
	\ifdefined\IsOutline\else
		\ifthenelse{\boolean{CHAPTERNOTES}}{%
			\theendnotes%
			\setcounter{endnote}{0}%
		}{
			\setcounter{endnote}{0}%
		}%
	\fi%
}

%%---------------------------------------------------------------------
\newcommand{\ChapterPrefix}{%
	\ifdefined\IsOutline\else%
		\ChapterSuffix%
		\clearpage%
	\fi%
}

%%---------------------------------------------------------------------
%% Redefine the \chapter command to include book characteristics
\let\originalchapter\chapter
\RenewDocumentCommand{\chapter}{ s o m }{% s:star, o:opional, m:mandatory
   \IfBooleanTF{#1}{%
      \originalchapter*{#3}%
   }{%
   	  \ifdefined\IsInBody\ChapterPrefix\else\def\IsInBody{true}\fi%
      \originalchapter{#3}%
      \ChapterNotesName{#3}%
   }%
}

%%---------------------------------------------------------------------
% Outputs the Book End Notes section, if needed.
\newcommand{\BookEndNotes}{%
	\ifthenelse{\boolean{CHAPTERNOTES}}{}{%
		\cleardoublepage\phantomsection%
		\addcontentsline{toc}{chapter}{Notes by Chapter}%
		\theendnotes%
	}%
}

%%---------------------------------------------------------------------
\newcommand{\LineSeparator}{%
    \begin{center}%
        \line(1,0){250}%
    \end{center}%
}

\newcommand{\MusicNote}{\twonotes\xspace}

%%---------------------------------------------------------------------
%% Format Specific notations - Identifies oprions set in output.
%%
\newcommand{\IntroSpec}{\ifthenelse{\boolean{NOBODY}}{IntroOnly\xspace}{}}%
\newcommand{\HtmlSpec}{\ifthenelse{\boolean{TOHTML}}{HTML\xspace}{}}%
\newcommand{\MobiSpec}{\ifthenelse{\boolean{TOMOBI}}{Mobi\xspace}{}}%
\newcommand{\EpubSpec}{\ifthenelse{\boolean{TOEPUB}}{Epub\xspace}{}}%
\newcommand{\MonoSpec}{\ifthenelse{\boolean{NOCOLOR}}{Grayscale\xspace}{}}%
\newcommand{\PressSpec}{\ifthenelse{\boolean{TOPRESS}}{PRINT\xspace}{}}%

\newcommand{\DraftSpec}{\ifthenelse{\boolean{FINALFORM}}{}%
{DRAFT--\HtmlSpec\MobiSpec\EpubSpec\MonoSpec\PressSpec\IntroSpec}}%

\newcommand{\FullDraftSpec}{\ifthenelse{\boolean{FINALFORM}}{}%
{ [-----\DraftSpec---]}}%

%%---------------------------------------------------------------------
%% HTML Specific Command overrides...
\ifthenelse{\boolean{TOHTML}}{%
    \renewcommand\LineSeparator{%
        \begin{center}%    
        \ifthenelse{\boolean{FINALFORM}}{%
             ---------------------------------
        }{%
             ---------\DraftSpec---------
	    }%
        \end{center}%
    }%
    \renewcommand{\MusicNote}{}%
}{}%

%%---------------------------------------------------------------------
\newcommand{\BigBold}[1]{
	\begin{flushleft}
	{\large{\textbf{--- #1 ---}}} \\
	\end{flushleft}
}

%%---------------------------------------------------------------------
% Define source directory for images as a parent directory or
% as a subdirectory.
\ifthenelse{\boolean{PARENTIMAGES}}{%
	\newcommand{\ImageSource}{../images}%
}{%
	\newcommand{\ImageSource}{images}%
}

%%---------------------------------------------------------------------
% Include a graphic image that selects _BW version as needed
% \Image[options]{name}{format}
\newcommand{\IMAGE}[2][]{%
	\ifthenelse{\boolean{FINALFORM}}{%
		\includegraphics[#1]{\ImageSource/#2}%
	}{%
		\ifthenelse{\boolean{NOIMAGES}}{%
			\Large{[ IMAGE:#2 ]}
		}{%
		    \IfFileExists{\ImageSource/#2}{%
			    \includegraphics[#1]{\ImageSource/#2}%
			}{%
				\Huge{MISSING IMAGE\\ \texttt{\ImageSource/#2}}\normalsize
			}%	
		}%
			
	}%
}

\newcommand{\Image}[3][]{%
    \ifthenelse{\boolean{NOCOLOR}}{%
	    \IfFileExists{\ImageSource/#2_BW.#3}{\IMAGE[#1]{#2_BW.#3}}{%
		    \IfFileExists{\ImageSource/#2-BW.#3}{\IMAGE[#1]{#2-BW.#3}}{%
			    \IfFileExists{\ImageSource/{#2}BW.#3}{\IMAGE[#1]{{#2}BW.#3}}{%
		    		\IMAGE[#1]{#2.#3}%
		    	}%
	    	}%
	    }%
	}{%
        \IMAGE[#1]{#2.#3}%
	}%
}

%%---------------------------------------------------------------------
% A column/page centered figure with caption.
% \CenterFigure{caption}{file}
\newcommand{\CenterFigure}[3]{
	\begin{figure}[ht]%
		\begin{center}%
			\Image[width=0.7\textwidth]{#2}{#3}%
			\caption{#1}%
		\end{center}%
	\end{figure}%
}

%%---------------------------------------------------------------------
\newcommand{\QuoteParagraph}[2]{
  \begin{adjustwidth}{1cm}{1cm}
  --- \sffamily#1\normalfont\ifthenelse{\isempty{#2}}{}{ --- #2}
  \end{adjustwidth}
}

%%---------------------------------------------------------------------
\newenvironment{boxedarea}
    {\begin{center}
    \begin{tabular}{|p{0.9\textwidth}|}
    \hline\\
    }
    { 
    \\\\\hline
    \end{tabular} 
    \end{center}
    }
   
%%---------------------------------------------------------------------
%% Set running headings at to of pages.
\newcommand{\RightRunningHeads}[1]{\renewcommand\rightmark{#1}}
\newcommand{\LeftRunningHeads}[1]{\renewcommand\leftmark{#1}}

\newcommand{\RunningHeads}[1]{
    \LeftRunningHeads{#1}%
    \RightRunningHeads{#1}%
}

%%--------------------------------------------------------------------
%% \noparskip in paragraph to suppress blank line after this one.
%%\let\svpar\par
%%\edef\svparskip{\the\parskip}
%%\def\revertpar{\svpar\setlength\parskip{\svparskip}\let\par\svpar}
%%\def\noparskip{\leavevmode\setlength\parskip{0pt}%
%%  \def\par{\svpar\let\par\revertpar}}

%--------------------------------------------------------------------
% LaTeX Processor independent environment variable access.
% \getenv[\targetname]{envvarname}
\ifxetex
  \usepackage{catchfile}
  \newcommand\getenv[2][]{%
    \immediate\write18{kpsewhich --var-value #2 > \jobname.tmp}%
    \CatchFileDef{\temp}{\jobname.tmp}{\endlinechar=-1}%
    \if\relax\detokenize{#1}\relax\temp\else\let#1\temp\fi}
\else
  \ifluatex
    \newcommand\getenv[2][]{%
      \edef\temp{\directlua{tex.sprint(
        kpse.var_value("\luatexluaescapestring{#2}") or "" ) }}%
      \if\relax\detokenize{#1}\relax\temp\else\let#1\temp\fi}
  \else
    \usepackage{catchfile}
    \newcommand{\getenv}[2][]{%
      \CatchFileEdef{\temp}{"|kpsewhich --var-value #2"}{\endlinechar=-1}%
      \if\relax\detokenize{#1}\relax\temp\else\let#1\temp\fi}
  \fi
\fi  
% End of initialization macros  
%--------------------------------------------------------------------

  